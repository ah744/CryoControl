The analysis will progress in stages. First an abstract model of a classical control unit as a cache is built, along with various methods for communicating program data to and from the cache. In this stage, program modules are extracted from full logical schedules which are compiled with the ScaffCC framework \cite{scaff}, and various cache compression techniques are explored for reducing the minimum required memory size of the control unit. Initial analysis will compare the effect of variable module compression techniques against the required computation introduced by performing the compression and decompression actions, which ultimately affects overall quantum program runtime. \par ~\par 
	The next stage is the construction of an energy budget cost model, which incorporates cooling capacity constraints of the different temperature interfaces of a dilution refrigerator with the effective energy usage per operation of different hardware materials. The cache model of classical control is expanded by introducing these new costs as constraints on computation, which affects the ability to perform module decompression and maintain the target temperature for a specified region. Techniques for optimizing module compression and decompression will be explored. Additionally, parameters describing latency caused by microwave links traversing thermal boundaries will be introduced as communication costs, which will be incorporated into the cost model.\par ~\par
	The following stage will begin to analyze the tradeoffs of different flattening thresholds for code modularization during compilation of quantum programs. Conceptually, flattening quantum benchmarks with a high gate count threshold will allow the compiler to be more efficient and reduce overall runtime, as more context is available during these optimizations. However this increases the size of each module, which increases the complexity and overhead of the caching mechanism. The interaction between these behaviors will be analyzed, and new techniques of both code modularization and module compression and decompression will be explored to optimize quantum program execution.\par ~\par 
		The ultimate goal of this analysis is to extract and quantify specific tradeoffs with system design choices involving control unit hardware, temperature level placement of control units, and memory sizes of control units. The aim is also to develop novel techniques for optimizing the transmission of large quantum programs through a control unit system, including optimized module compression and decompression methods for both control unit memory size reduction and communication bandwidth requirement reduction. 

