A unique characteristic of quantum programs is a large amount of static compilation. Knowing this, quantum programs can be compiled in advance, and large amounts of optimization can be performed. Of interest to systems under the aforementioned hardware constraints are the tradeoffs between code flattening, parallelism extraction, and generated code size. 

Due to the size of some large quantum programs, full code flattening for parallelism extraction can be rendered intractible or infeasible, given classical compiler resource constraints. Additionally, selectively flattening large quantum programs can generate flattened code modules of different size distributions, and will extract varying levels of parallelism from the application altogether. Of interest to this study is the application of specific flattening mechanisms, the parallelism that these mechanisms are able to extract, and the tradeoffs present within extracting this parallelism at the cost of a specific flattened module size distribution. One flattening mechanism may extract large amounts of parallelism, but may create modules of sizes that generate poor caching behavior by the classical control units located within the cryogenic system.


